%%%%%%%%%%%%%%%%%%%%%%%%%%%%%%%%%%%%%%%%%%%%%%%%%%%%%%%%%%%%%%%%%%%%%%%%%%%%%%
% geDIG: 多層評価による洞察生成フレームワーク
% 改訂版 v4.0 (2025/08/03)
%%%%%%%%%%%%%%%%%%%%%%%%%%%%%%%%%%%%%%%%%%%%%%%%%%%%%%%%%%%%%%%%%%%%%%%%%%%%%%
\documentclass[uplatex]{bxjsarticle}
\usepackage{amsmath,amssymb,graphicx,algorithm,algorithmic,booktabs}
\usepackage{hyperref}
\usepackage{float}
\usepackage[whole]{bxcjkatype} % 日本語文字対応
\title{geDIG: 多層評価による洞察生成フレームワーク\\
\large{〜脳の階層的処理に着想を得た知識グラフの動的再編成〜}}
\author{宮内和義\thanks{independent researcher, \texttt{email@example.com}}}
\date{2025年8月3日}

\begin{document}
\maketitle

%====================================================================
\begin{abstract}
本研究では、脳の階層的情報処理に着想を得て、知識グラフの\textbf{多層構造評価}による
洞察生成フレームワーク\emph{geDIG}を提案する。
通常のクエリ応答では局所的な最適化(1-hop)により予測誤差最小化を実現し、
新規性の高いクエリに対してはx-hop先までの評価により、
ノード・エッジ追加にも関わらずグラフ編集距離(GED)が低下する
「構造的洞察」を検出する。
GED低下箇所のノード間でメッセージパッシングを行うことで、
新たな概念(洞察ベクトル)を生成する。
迷路実験では成功率96.7\%(PPO比較)、
質問応答実験では最小知識選択でF1スコア0.89を達成し、
本手法の有効性を実証した。
\end{abstract}

%====================================================================
\section{はじめに}

人工知能における「洞察(insight)」の獲得は、
単なる情報の蓄積ではなく、知識構造の質的な転換を伴う。
既存の機械学習アプローチは学習と推論を分離したフェーズとして扱い、
推論時の動的な知識再編成を捉えきれていない。

本研究は、脳の海馬における階層的な情報処理機構に着想を得て、
以下の統合的なフレームワークを提案する:

\begin{enumerate}
\item \textbf{局所最適化}: 1-hop評価によるクエリ応答(予測誤差最小化)
\item \textbf{大域的評価}: x-hop評価による構造的洞察の検出
\item \textbf{概念生成}: メッセージパッシングによる新概念の創発
\end{enumerate}

%====================================================================
\section{脳の階層処理とgeDIG}

\subsection{理論的基盤}

脳の情報処理は、局所的な予測と大域的な統合の二重構造を持つ。
海馬のplace cellsは局所的な位置情報を符号化し、
grid cellsはより広域な空間構造を表現する\cite{moser2008}。
この階層性を知識グラフ処理に応用する。

\subsection{geDIG報酬関数}

知識グラフ$\mathcal{G}_t=(\mathcal{V}_t,\mathcal{E}_t)$の
時刻$t$から$t+1$への変化に対する報酬:

\begin{equation}
\mathcal{F}_t = w \cdot \Delta\mathrm{GED}_t - kT \cdot \Delta\mathrm{IG}_t
\label{eq:gedig}
\end{equation}

ここで:
\begin{itemize}
\item $\Delta\mathrm{GED}_t$: グラフ編集距離の変化
\item $\Delta\mathrm{IG}_t$: 情報利得(エントロピー減少)
\item $w, kT$: 重み係数
\end{itemize}

%====================================================================
\section{多層評価メカニズム}

\subsection{1-hop評価:局所最適化}

通常のクエリ$q$に対して、1-hop近傍での最適化:

\begin{equation}
\mathcal{N}_1(q) = \arg\min_{v \in \mathcal{V}} \mathcal{F}(q, v)
\label{eq:1hop}
\end{equation}

これは予測誤差最小化と等価であり、
効率的なクエリ応答を実現する。

\subsection{x-hop評価:構造的洞察の検出}

新規性の高いクエリに対して、x-hop($x \geq 2$)での評価:

\begin{equation}
\Delta\mathrm{GED}_x = \mathrm{GED}(\mathcal{G}_x^{before}, \mathcal{G}_x^{after})
\label{eq:xhop}
\end{equation}

重要な発見:
\begin{theorem}[構造的洞察の条件]
クエリノード$q$の追加により$|\mathcal{E}|$が増加するにも関わらず、
$\Delta\mathrm{GED}_x < 0$となる場合が存在する。
これは$q$が離散的な概念群を統合する「橋渡し」として機能する時に生じる。
\end{theorem}

\subsection{洞察ベクトルの生成}

GED低下が検出された箇所$\{v_i, v_j\}$において、
メッセージパッシングにより新概念を生成:

\begin{equation}
h_{insight} = \sigma\left(\sum_{k \in \{i,j,q\}} W_k h_k + b\right)
\label{eq:insight}
\end{equation}

%====================================================================
\section{実験と結果}

\subsection{実験1:迷路ナビゲーション}

\paragraph{設定}
15×15迷路、100エピソード、視覚情報あり。
PPO(200エピソード学習後)との比較。

\paragraph{結果}
\begin{itemize}
\item geDIG成功率: \textbf{96.7\%}(ゼロショット)
\item PPO成功率: 47.0\%(200エピソード学習後)
\item 平均ステップ数: 37.7 vs 281.5
\end{itemize}

\paragraph{2-hop評価の効果}
分岐点での判断において、1-hopでは局所最適に陥るケースでも、
2-hop評価により「戻った方が良い」という判断が可能になった。
これは以下のクエリ設計により実現:

\begin{algorithmic}
\STATE \textbf{if} 行き止まり検出 \textbf{then}
\STATE \quad $q \leftarrow$ "この先に目標はあるか?"
\STATE \quad $eval\_2hop(current\_position, q)$
\STATE \quad \textbf{if} $\Delta\mathrm{GED}_2 < threshold$ \textbf{then}
\STATE \quad \quad バックトラック実行
\STATE \quad \textbf{end if}
\STATE \textbf{end if}
\end{algorithmic}

\subsection{実験2:質問応答(最小知識選択)}

\paragraph{設定}
知識エントリ10件、質問5件。
geDIGによる最小知識選択 vs 全知識使用。

\paragraph{結果}
\begin{itemize}
\item 平均選択知識数: 2.0(最小限)
\item F1スコア: 0.89
\item 処理時間: <0.1秒/質問
\end{itemize}

\paragraph{洞察生成例}
質問「なぜ虹は七色か」に対して:
\begin{enumerate}
\item 1-hop: \{光, 屈折, 色\}を選択
\item 2-hop評価: $\Delta\mathrm{GED}_2 = -0.8$(スパイク検出)
\item GED低下箇所: (光, 色)間
\item 生成された洞察: 「光の分散現象」
\end{enumerate}

%====================================================================
\section{議論}

\subsection{なぜ2-hopでGEDが低下するのか}

1-hopでのノード追加は局所的にはグラフを複雑化する。
しかし、2-hop視点では以下のメカニズムでGEDが低下:

\begin{enumerate}
\item \textbf{ショートカット効果}: 遠回りしていた経路が短縮
\item \textbf{ハブ形成}: 分散していた接続が集約
\item \textbf{冗長性削減}: 重複した経路が統合
\end{enumerate}

\subsection{脳との対応}

提案手法は以下の脳機能と対応:
\begin{itemize}
\item 1-hop最適化 $\leftrightarrow$ 皮質の局所処理
\item x-hop評価 $\leftrightarrow$ 海馬の大域的統合
\item メッセージパッシング $\leftrightarrow$ シナプス伝達
\end{itemize}

\subsection{限界と今後の課題}

\begin{enumerate}
\item 計算量: x-hopの増加に伴いO($n^x$)で増大
\item パラメータ調整: $w, kT$の自動調整機構が必要
\item スケーラビリティ: 大規模知識グラフへの適用
\end{enumerate}

%====================================================================
\section{関連研究}

\paragraph{内発的動機づけ}
Curiosity-driven learning\cite{pathak2017}は
予測誤差を報酬とするが、構造変化は考慮しない。

\paragraph{グラフニューラルネットワーク}
GNN\cite{kipf2017}はメッセージパッシングを用いるが、
動的なグラフ編集は扱わない。

\paragraph{神経科学}
海馬のリプレイ機構\cite{foster2006}は
経験の再編成を行うが、明示的な構造評価はない。

%====================================================================
\section{結論}

本研究では、脳の階層的処理に着想を得たgeDIGフレームワークを提案した。
主要な貢献:

\begin{enumerate}
\item \textbf{理論的貢献}: 1-hop局所最適化とx-hop大域評価の統合
\item \textbf{技術的貢献}: GED低下検出による洞察生成メカニズム
\item \textbf{実証的貢献}: 迷路実験96.7\%、質問応答F1=0.89
\end{enumerate}

geDIGは「学習しながら推論し、推論しながら学習する」
新しいAIパラダイムの基礎となる。

%====================================================================
\begin{thebibliography}{99}
\bibitem{moser2008} Moser, E. I., Kropff, E., \& Moser, M. B. (2008). 
Place cells, grid cells, and the brain's spatial representation system. 
Annual review of neuroscience, 31, 69-89.

\bibitem{pathak2017} Pathak, D., Agrawal, P., Efros, A. A., \& Darrell, T. (2017). 
Curiosity-driven exploration by self-supervised prediction. 
In ICML.

\bibitem{foster2006} Foster, D. J., \& Wilson, M. A. (2006). 
Reverse replay of behavioural sequences in hippocampal place cells during the awake state. 
Nature, 440(7084), 680-683.

\bibitem{kipf2017} Kipf, T. N., \& Welling, M. (2017). 
Semi-supervised classification with graph convolutional networks. 
In ICLR.
\end{thebibliography}

\end{document}