%%%%%%%%%%%%%%%%%%%%%%%%%%%%%%%%%%%%%%%%%%%%%%%%%%%%%%%%%%%%%%%%%%%%%%%%%%%%%%
% geDIG Paper Appendix: Detailed Results and Visualizations
% 日本語フルテキスト版 付録  (2025/07/23)
%%%%%%%%%%%%%%%%%%%%%%%%%%%%%%%%%%%%%%%%%%%%%%%%%%%%%%%%%%%%%%%%%%%%%%%%%%%%%%
\documentclass[uplatex]{bxjsarticle}
\usepackage{amsmath,amssymb,graphicx,algorithm,algorithmic,booktabs}
\usepackage{hyperref}
\usepackage{float}
\usepackage{subcaption}
\usepackage{multirow}

\title{付録:geDIG実験の詳細結果と可視化}
\author{宮内一義}
\date{2025年7月23日}

\begin{document}
\maketitle

%====================================================================
\section{A. 知識ベース全100項目}

本実験で使用した階層的知識ベースの全100項目を示す。
各項目は5つのフェーズに分類されている:
\begin{itemize}
  \item Phase 1 (基礎): 基本概念の定義(10項目)
  \item Phase 2 (関係): 概念間の関係性(20項目)
  \item Phase 3 (統合): 統合的理解(20項目)
  \item Phase 4 (探索): 探索的問い(10項目)
  \item Phase 5 (超越): 洞察と推測(40項目)
\end{itemize}

\begin{table}[H]
\centering
\caption{知識ベース全100項目(Phase別)}
\scriptsize
\begin{tabular}{|c|c|l|p{9cm}|}
\hline
\multicolumn{4}{|c|}{\textbf{Phase 1: Foundational (基礎概念)}} \\
\hline
ID & Phase & Category & Knowledge Item \\
\hline
1 & 1 & physics & Energy is the capacity to do work and exists in various forms. \\
2 & 1 & information\_theory & Information is the reduction of uncertainty in a system. \\
3 & 1 & thermodynamics & Entropy measures the disorder or randomness in a system. \\
4 & 1 & ai & Neural networks process information through weighted connections. \\
5 & 1 & biology & DNA encodes genetic information in a four-letter alphabet. \\
6 & 1 & quantum & Quantum states can exist in superposition until measured. \\
7 & 1 & neuroscience & Consciousness involves integrated information processing. \\
8 & 1 & biology & Evolution optimizes organisms through natural selection. \\
9 & 1 & physics & Black holes have maximum entropy for their size. \\
10 & 1 & computing & Computation requires minimum energy per bit erased. \\
\hline
\multicolumn{4}{|c|}{\textbf{Phase 2: Relational (概念間の関係)}} \\
\hline
11 & 2 & interdisciplinary & Shannon entropy and thermodynamic entropy share mathematical structure. \\
12 & 2 & biology & Living systems decrease local entropy by increasing global entropy. \\
13 & 2 & quantum & Quantum entanglement creates non-local correlations between particles. \\
14 & 2 & neuroscience & Neural plasticity allows brains to reorganize and form new connections. \\
15 & 2 & quantum & Information cannot be destroyed in quantum mechanics. \\
16 & 2 & ai & Machine learning extracts patterns from high-dimensional data. \\
17 & 2 & ai & Genetic algorithms mimic evolutionary processes for optimization. \\
18 & 2 & physics & Thermodynamic arrow of time emerges from entropy increase. \\
19 & 2 & neuroscience & Synaptic pruning optimizes brain connectivity during development. \\
20 & 2 & physics & Landauer's principle links information erasure to heat generation. \\
21 & 2 & physics & Maxwell's demon thought experiment connects information and thermodynamics. \\
22 & 2 & biology & Protein folding minimizes free energy to find stable configurations. \\
23 & 2 & physics & Holographic principle suggests information is encoded on boundaries. \\
24 & 2 & ai & Attention mechanisms in AI focus computational resources selectively. \\
25 & 2 & biology & Epigenetics shows how environment influences gene expression. \\
26 & 2 & quantum & Wave function collapse may be an information update process. \\
27 & 2 & complexity & Emergence occurs when simple rules produce complex behaviors. \\
28 & 2 & systems & Feedback loops can amplify or dampen system responses. \\
29 & 2 & physics & Phase transitions mark qualitative changes in system behavior. \\
30 & 2 & complexity & Self-organization creates order without external control. \\
\hline
\end{tabular}
\end{table}

% Continue table on next page
\begin{table}[H]
\centering
\scriptsize
\begin{tabular}{|c|c|l|p{9cm}|}
\hline
\multicolumn{4}{|c|}{\textbf{Phase 3: Integrative (統合的理解)}} \\
\hline
ID & Phase & Category & Knowledge Item \\
\hline
31 & 3 & interdisciplinary & Energy, information, and entropy form a fundamental trinity in physics. \\
32 & 3 & interdisciplinary & Life can be viewed as a far-from-equilibrium dissipative structure. \\
33 & 3 & interdisciplinary & Consciousness might emerge from integrated information processing. \\
34 & 3 & interdisciplinary & Evolution can be understood as an information optimization process. \\
35 & 3 & interdisciplinary & Quantum computing exploits superposition for parallel processing. \\
36 & 3 & interdisciplinary & The universe might be fundamentally computational in nature. \\
37 & 3 & interdisciplinary & Intelligence involves prediction, compression, and pattern recognition. \\
38 & 3 & cognitive & Creativity emerges from novel connections between existing concepts. \\
39 & 3 & neuroscience & Memory consolidation involves replay and reorganization of experiences. \\
40 & 3 & cognitive & Language enables infinite expression from finite components. \\
41 & 3 & mathematics & Fractals show self-similarity across different scales. \\
42 & 3 & mathematics & Chaos theory reveals sensitive dependence on initial conditions. \\
43 & 3 & systems & Network effects create value through increased connectivity. \\
44 & 3 & physics & Symmetry breaking leads to diversity in physical systems. \\
45 & 3 & physics & Renormalization reveals universal behaviors across scales. \\
46 & 3 & biology & Autopoiesis describes self-creating and self-maintaining systems. \\
47 & 3 & systems & Stigmergy enables indirect coordination through environmental modification. \\
48 & 3 & ai & Swarm intelligence emerges from simple local interactions. \\
49 & 3 & cognitive & Metacognition involves thinking about thinking processes. \\
50 & 3 & cognitive & Embodied cognition suggests mind emerges from body-environment interaction. \\
\hline
\multicolumn{4}{|c|}{\textbf{Phase 4: Exploratory (探索的問い)}} \\
\hline
51 & 4 & question & Can consciousness be quantified through integrated information theory? \\
52 & 4 & question & Is the universe a quantum computer processing information? \\
53 & 4 & question & How does quantum mechanics relate to consciousness? \\
54 & 4 & question & Can artificial systems achieve genuine understanding? \\
55 & 4 & question & What is the relationship between computation and physics? \\
56 & 4 & question & How does complexity emerge from simplicity? \\
57 & 4 & question & Is information more fundamental than matter and energy? \\
58 & 4 & question & Can evolution be directed through information design? \\
59 & 4 & question & What are the limits of predictability in complex systems? \\
60 & 4 & question & How does meaning emerge from syntax in language and thought? \\
\hline
\end{tabular}
\end{table}

% Final part of table
\begin{table}[H]
\centering
\scriptsize
\begin{tabular}{|c|c|l|p{9cm}|}
\hline
\multicolumn{4}{|c|}{\textbf{Phase 5: Transcendent (洞察と超越)}} \\
\hline
ID & Phase & Category & Knowledge Item \\
\hline
61 & 5 & insight & The observer and observed are fundamentally inseparable in quantum mechanics. \\
62 & 5 & insight & All physical laws might reduce to information conservation principles. \\
63 & 5 & insight & Consciousness could be the universe's way of observing itself. \\
64 & 5 & insight & Intelligence is compression - finding minimal descriptions of reality. \\
65 & 5 & insight & Time might emerge from entanglement and information flow. \\
66 & 5 & insight & Life accelerates entropy production while creating local order. \\
67 & 5 & insight & Creativity is the universe exploring its possibility space. \\
68 & 5 & insight & Memory is time travel through information preservation. \\
69 & 5 & insight & Language is collective consciousness encoded in symbols. \\
70 & 5 & insight & Mathematics is the universe's self-description language. \\
71 & 5 & speculation & Free will might emerge from quantum indeterminacy amplified by chaos. \\
72 & 5 & speculation & Dark matter could be information structures beyond standard physics. \\
73 & 5 & speculation & Dreams might be consciousness exploring counterfactual realities. \\
74 & 5 & cognitive & Intuition accesses implicit patterns below conscious threshold. \\
75 & 5 & speculation & Synchronicity reveals hidden causal connections in reality. \\
76 & 5 & systems & Collective intelligence transcends individual cognitive limits. \\
77 & 5 & interdisciplinary & Technology extends human cognition into the environment. \\
78 & 5 & interdisciplinary & Culture is distributed cognition across minds and time. \\
79 & 5 & cognitive & Art explores the space between order and chaos. \\
80 & 5 & philosophy & Science is systematic curiosity encoded in methodology. \\
81 & 5 & cognitive & Meditation might access fundamental computational substrates of mind. \\
82 & 5 & neuroscience & Psychedelics could reveal alternative information processing modes. \\
83 & 5 & cognitive & Music is emotional information encoded in temporal patterns. \\
84 & 5 & speculation & Love might be recognition of deep informational resonance. \\
85 & 5 & philosophy & Death is information pattern dissolution and transformation. \\
86 & 5 & technology & Virtual reality creates new experiential possibility spaces. \\
87 & 5 & technology & Blockchain enables trustless information consensus. \\
88 & 5 & technology & Internet forms a global nervous system for humanity. \\
89 & 5 & ai & AI might discover alien forms of intelligence and consciousness. \\
90 & 5 & technology & Biotechnology enables conscious evolution and design. \\
91 & 5 & technology & Nanotechnology bridges quantum and classical information realms. \\
92 & 5 & philosophy & Space exploration extends consciousness beyond Earth. \\
93 & 5 & systems & Sustainability requires balancing information and energy flows. \\
94 & 5 & systems & Democracy is distributed information processing for governance. \\
95 & 5 & systems & Economics tracks information about value and scarcity. \\
96 & 5 & systems & Education is intergenerational information transfer and transformation. \\
97 & 5 & philosophy & Ethics emerges from recognition of shared information patterns. \\
98 & 5 & philosophy & Beauty might signal deep informational coherence and harmony. \\
99 & 5 & cognitive & Wisdom integrates knowledge across multiple domains and timescales. \\
100 & 5 & philosophy & The future is the universe computing its next state from present information. \\
\hline
\end{tabular}
\label{tab:kb_full}
\end{table}

%====================================================================
\section{B. 実験1: 大規模洞察検出実験の詳細結果}

\subsection{B.1 質問別パフォーマンス}

20の評価質問すべてに対する詳細な結果を示す。

\begin{table}[H]
\centering
\caption{評価質問全20問の結果}
\scriptsize
\begin{tabular}{|c|p{7cm}|l|c|c|c|c|}
\hline
ID & Question & Type & Diff. & Spike & Conf. & Time(s) \\
\hline
1 & How does information theory relate to thermodynamics? & conceptual\_integration & Med & ✓ & 0.845 & 0.051 \\
2 & Can consciousness emerge from quantum processes? & speculative\_integration & Hard & ✓ & 0.916 & 0.039 \\
3 & How do living systems maintain order against entropy? & systems\_thinking & Med & ✓ & 0.861 & 0.046 \\
4 & What is the relationship between evolution and information? & interdisciplinary & Med & ✓ & 0.912 & 0.042 \\
5 & How does the brain process and integrate information? & neuroscience & Med & ✓ & 0.950 & 0.036 \\
6 & Can artificial intelligence achieve genuine creativity? & ai\_philosophy & Hard & ✓ & 0.800 & 0.046 \\
7 & What is the fundamental nature of reality - matter, energy, or information? & foundational & Hard & ✓ & 0.995 & 0.069 \\
8 & How does complexity emerge from simple rules? & complexity\_science & Med & ✓ & 0.635 & 0.041 \\
9 & What role does observation play in quantum mechanics? & quantum\_foundations & Med & ✓ & 0.885 & 0.050 \\
10 & How does memory consolidation work in biological systems? & neuroscience & Easy & ✓ & 0.779 & 0.039 \\
11 & Can the universe be understood as a computational system? & computational\_universe & Hard & ✓ & 0.686 & 0.039 \\
12 & How do feedback loops influence system behavior? & systems\_dynamics & Easy & ✗ & 0.408 & 0.035 \\
13 & What is the connection between language and consciousness? & cognitive\_science & Med & ✓ & 0.865 & 0.041 \\
14 & How does genetic information translate into biological form? & biology & Easy & ✓ & 0.825 & 0.042 \\
15 & Can collective intelligence surpass individual cognition? & collective\_systems & Med & ✗ & 0.487 & 0.034 \\
16 & What is the relationship between chaos and predictability? & dynamical\_systems & Med & ✓ & 0.815 & 0.038 \\
17 & How does technology extend human cognitive capabilities? & human\_technology & Easy & ✓ & 0.829 & 0.043 \\
18 & What principles govern self-organizing systems? & emergence & Med & ✗ & 0.487 & 0.036 \\
19 & How might quantum computing revolutionize information processing? & quantum\_computing & Med & ✓ & 0.885 & 0.042 \\
20 & What is the role of information in the arrow of time? & physics\_philosophy & Hard & ✓ & 0.836 & 0.100 \\
\hline
\multicolumn{4}{|c|}{\textbf{Summary}} & 17/20 & 0.841* & 0.045 \\
\hline
\end{tabular}
\label{tab:questions_all}
\end{table}

\textit{*平均確信度はスパイク検出された質問のみの平均}

\subsection{B.2 グラフ構造の経時変化}

質問ごとのグラフ構造の変化を追跡した結果を示す。

\begin{table}[H]
\centering
\caption{グラフメトリクスの経時変化}
\scriptsize
\begin{tabular}{|c|c|c|c|c|c|c|}
\hline
Q & Nodes & Edges & Avg Path & Clustering & Connectivity & Phase Div \\
\hline
0 & 100 & 962 & 2.8 & 0.42 & - & - \\
1 & 105 & 985 & 2.7 & 0.44 & 0.76 & 0.80 \\
2 & 108 & 1003 & 2.5 & 0.46 & 0.84 & 0.90 \\
3 & 111 & 1018 & 2.6 & 0.45 & 0.81 & 0.85 \\
4 & 114 & 1035 & 2.5 & 0.47 & 0.88 & 0.88 \\
5 & 117 & 1052 & 2.4 & 0.48 & 0.92 & 0.95 \\
6 & 119 & 1068 & 2.3 & 0.49 & 0.86 & 0.80 \\
7 & 122 & 1089 & 1.9 & 0.52 & 0.96 & 1.00 \\
8 & 124 & 1098 & 2.2 & 0.50 & 0.71 & 0.70 \\
9 & 126 & 1112 & 2.1 & 0.51 & 0.87 & 0.85 \\
10 & 128 & 1124 & 2.2 & 0.50 & 0.80 & 0.75 \\
11 & 130 & 1138 & 2.1 & 0.51 & 0.75 & 0.90 \\
12 & 131 & 1142 & 2.1 & 0.51 & 0.45 & 0.50 \\
13 & 133 & 1156 & 2.0 & 0.52 & 0.85 & 0.88 \\
14 & 135 & 1168 & 2.0 & 0.52 & 0.82 & 0.70 \\
15 & 136 & 1173 & 2.0 & 0.52 & 0.48 & 0.55 \\
16 & 138 & 1185 & 1.9 & 0.53 & 0.84 & 0.82 \\
17 & 140 & 1197 & 2.0 & 0.52 & 0.81 & 0.73 \\
18 & 141 & 1202 & 2.0 & 0.52 & 0.49 & 0.60 \\
19 & 143 & 1216 & 1.9 & 0.53 & 0.87 & 0.86 \\
20 & 145 & 1232 & 1.8 & 0.54 & 0.88 & 0.92 \\
\hline
\end{tabular}
\label{tab:graph_evolution}
\end{table}

特に注目すべきは、質問7(「現実の本質」)で平均経路長が2.3から1.9に急減し、
クラスタリング係数が0.49から0.52に上昇した点である。
これはグラフの大規模な再編成を示唆している。

\subsection{B.3 応答比較}

代表的な質問に対する3手法の応答を詳細に比較する。

\subsubsection{質問1: How does information theory relate to thermodynamics?}

\begin{table}[H]
\centering
\caption{応答比較 - 情報理論と熱力学の関係}
\small
\begin{tabular}{|p{3cm}|p{11cm}|}
\hline
\textbf{手法} & \textbf{応答} \\
\hline
Baseline LLM & I need more context to answer this question about information theory and thermodynamics. \\
\hline
RAG & According to the knowledge base, Shannon entropy and thermodynamic entropy share mathematical structure, and Maxwell's demon connects information and thermodynamics. \\
\hline
InsightSpike (geDIG) & This question bridges multiple conceptual levels:\newline
\textbf{Foundational}: Information is the reduction of uncertainty in a system..., Entropy measures the disorder or randomness in a system...\newline
\textbf{Relational}: Maxwell's demon thought experiment connects information and thermodynamics..., Shannon entropy and thermodynamic entropy share mathematical structure...\newline
\textbf{Integrative}: Energy, information, and entropy form a fundamental trinity...\newline
\textbf{Exploratory}: Is information more fundamental than matter and energy?...\newline
The high connectivity (ratio: 0.76) and phase diversity (0.80) suggest an emergent insight at the intersection of these concepts. \\
\hline
\end{tabular}
\end{table}

\subsubsection{質問7: What is the fundamental nature of reality?}

\begin{table}[H]
\centering
\caption{応答比較 - 現実の本質}
\small
\begin{tabular}{|p{3cm}|p{11cm}|}
\hline
\textbf{手法} & \textbf{応答} \\
\hline
Baseline LLM & Reality could be made of matter, energy, or information - this is still debated. \\
\hline
RAG & The knowledge base indicates that energy, information, and entropy form a fundamental trinity, with all physical laws potentially reducing to information conservation. \\
\hline
InsightSpike (geDIG) & This question bridges multiple conceptual levels:\newline
\textbf{Foundational}: Information is the reduction of uncertainty...\newline
\textbf{Relational}: Maxwell's demon thought experiment connects information...\newline
\textbf{Integrative}: Energy, information, and entropy form a fundamental trinity...\newline
\textbf{Exploratory}: Is information more fundamental than matter and energy?...\newline
\textbf{Transcendent}: Intelligence is compression - finding minimal descriptions..., Dark matter could be information structures beyond...\newline
The high connectivity (ratio: 0.96) and phase diversity (1.00) suggest an emergent insight at the intersection of these concepts. \\
\hline
\end{tabular}
\end{table}

%====================================================================
\section{C. 知識グラフの可視化}

\subsection{C.1 クエリ注入による知識グラフの進化}

図\ref{fig:knowledge_graph}は、質問クエリが知識グラフに注入された際の構造変化を示す。
ベースグラフ(左上)から、各質問に対して関連ノードが活性化され(赤色)、
ノード間の接続パターンが変化する様子が観察できる。

\begin{figure}[H]
\centering
\includegraphics[width=\textwidth]{knowledge_graph_query_injection.png}
\caption{知識グラフへのクエリ注入と構造変化。赤いノードは質問に関連する知識項目を示し、
赤い太線は活性化された接続を表す。SPIKEラベルは洞察検出を示す。}
\label{fig:knowledge_graph}
\end{figure}

\subsection{C.2 スパイク検出メカニズム}

図\ref{fig:spike_mechanism}は、geDIGのスパイク検出メカニズムを可視化したものである。

\begin{figure}[H]
\centering
\includegraphics[width=\textwidth]{spike_detection_mechanism.png}
\caption{(左) グラフメトリクスの時間発展。接続性比率とフェーズ多様性が閾値を超えた時点でスパイクが検出される。
(右) geDIG報酬関数の景観。最適な洞察パスは構造変化(ΔGED)と情報利得(ΔIG)のバランスを取る。}
\label{fig:spike_mechanism}
\end{figure}

%====================================================================
\section{D. 統計的分析}

\subsection{D.1 実験サマリー統計}

\begin{table}[H]
\centering
\caption{実験統計サマリー}
\begin{tabular}{|l|r|}
\hline
Metric & Value \\
\hline
Total Questions & 20 \\
Spike Detection Rate & 85.0\% \\
Avg Processing Time & 45ms \\
Graph Nodes & 100 \\
Graph Edges & 962 \\
Avg Confidence (Spikes) & 0.841 \\
\hline
\end{tabular}
\label{tab:summary_stats}
\end{table}

\subsection{D.2 難易度別分析}

興味深いことに、質問の難易度が上がるほどスパイク検出率が向上する「難易度逆転現象」が観察された:

\begin{itemize}
  \item Easy: 75.0\% (3/4)
  \item Medium: 81.8\% (9/11)  
  \item Hard: 100\% (5/5)
\end{itemize}

これは、geDIGが単純な事実検索よりも複雑な概念統合に適していることを示唆している。

%====================================================================
\section{E. 実装の詳細}

\subsection{E.1 ハイパーパラメータ}

\begin{table}[H]
\centering
\caption{主要ハイパーパラメータ}
\begin{tabular}{|l|c|l|}
\hline
Parameter & Value & Description \\
\hline
Embedding Model & all-MiniLM-L6-v2 & Sentence-BERT model \\
Similarity Threshold & 0.3-0.4 & Phase-dependent \\
Max Relevant Nodes & 10 & Per query \\
Spike Threshold & 0.5 & For detection \\
Temperature (kT) & 1.0 & Exploration-exploitation \\
\hline
\end{tabular}
\end{table}

\subsection{E.2 計算複雑性}

- ノード追加: O(n) where n = existing nodes
- クエリ処理: O(n) for embedding comparison
- スパイク検出: O(k²) where k = relevant nodes (k << n)

全体として、線形以下のスケーリング特性を実現している。

%====================================================================
\section{F. 再現性}

実験の完全な再現のため、以下のリソースを提供:

\begin{itemize}
  \item ソースコード: \url{https://github.com/miyauchikun/InsightSpike-AI}
  \item 知識ベース: \texttt{experiments/comprehensive\_gedig\_evaluation/data/}
  \item 事前学習モデル: HuggingFace経由で自動ダウンロード
  \item 実行環境: Python 3.8+, PyTorch 1.9+
\end{itemize}

実行コマンド:
\begin{verbatim}
cd experiments/comprehensive_gedig_evaluation
poetry install
poetry run python src/comprehensive_gedig_experiment.py
\end{verbatim}

%====================================================================
\section{G. 実験2: 数学概念進化実験の詳細}

\subsection{G.1 概念進化の時系列データ}

学習段階ごとに投入された20の数学概念と、それに伴うグラフ構造の変化を示す。

\begin{table}[H]
\centering
\caption{数学概念の段階的学習過程}
\scriptsize
\begin{tabular}{|c|c|p{6cm}|c|c|c|c|}
\hline
ID & Phase & 概念 & ΔGED & ΔIG & $\mathcal{F}_t$ & 記憶操作 \\
\hline
\multicolumn{7}{|c|}{\textbf{Phase 1: 小学校レベル}} \\
\hline
1 & 1 & 分数はpizzaを切ったもの & -0.2 & 0.1 & 0.15 & 新規 \\
2 & 1 & 掛け算は同じ数を何回も足すこと & -0.1 & 0.1 & 0.10 & 新規 \\
3 & 1 & 0で割ることはできない(理由は不明) & -0.3 & 0.2 & 0.25 & 新規 \\
4 & 1 & マイナス×マイナスは謎だけどプラス & -0.4 & 0.3 & 0.35 & 新規 \\
5 & 1 & 面積はたて×よこ & -0.1 & 0.1 & 0.10 & 新規 \\
\hline
\multicolumn{7}{|c|}{\textbf{Phase 2: 中学校レベル}} \\
\hline
6 & 2 & 分数は比を表す & -1.8 & 0.6 & \textbf{2.1} & \textbf{統合} \\
7 & 2 & 負の数×負の数=正の数(数直線で理解) & -0.5 & 0.4 & 0.45 & 拡張 \\
8 & 2 & 文字式で一般化できる & -0.6 & 0.5 & 0.55 & 新規 \\
9 & 2 & 平方根は2乗したらその数になる数 & -0.3 & 0.3 & 0.30 & 新規 \\
10 & 2 & 方程式は釣り合いを表す & -0.4 & 0.4 & 0.40 & 新規 \\
\hline
\multicolumn{7}{|c|}{\textbf{Phase 3: 高校レベル}} \\
\hline
11 & 3 & 関数は写像である & -0.7 & 0.5 & 0.65 & 拡張 \\
12 & 3 & 微分は瞬間の変化率 & -0.8 & 0.6 & 0.70 & 新規 \\
13 & 3 & 積分は微分の逆演算 & -2.3 & 0.8 & \textbf{2.7} & \textbf{分裂} \\
14 & 3 & 複素数 i は2乗すると-1 & -0.9 & 0.7 & 0.85 & 新規 \\
15 & 3 & ベクトルは大きさと方向を持つ & -0.5 & 0.4 & 0.45 & 新規 \\
\hline
\multicolumn{7}{|c|}{\textbf{Phase 4: 大学レベル}} \\
\hline
16 & 4 & 実数は完備順序体 & -3.1 & 1.0 & \textbf{3.5} & \textbf{再編成} \\
17 & 4 & 関数は集合間の対応 & -0.8 & 0.6 & 0.70 & 再定義 \\
18 & 4 & 積分はルベーグ測度による & -1.2 & 0.8 & 1.00 & 拡張 \\
19 & 4 & 0で割るは未定義ではなく、拡張可能 & -1.5 & 0.9 & 1.35 & 再定義 \\
20 & 4 & カテゴリー論による数学の統一 & -2.0 & 1.0 & 1.90 & 統合 \\
\hline
\end{tabular}
\label{tab:math_evolution}
\end{table}

\subsection{G.2 グラフ構造の変化}

\begin{figure}[H]
\centering
\includegraphics[width=\textwidth]{figures/math_concept_graph_evolution.png}
\caption{数学概念学習に伴う知識グラフの構造変化。
(左)初期状態:分散した概念、
(中)統合・分裂期:ハブの形成、
(右)再編成後:階層的構造の出現}
\label{fig:math_graph_evolution}
\end{figure}

\subsection{G.3 記憶操作の詳細分析}

\subsubsection{統合操作(Merge)}

\textbf{例: 分数概念の統合(ID 6)}
\begin{itemize}
\item 前: 「pizza分数」と「比分数」が別個のエピソード
\item 後: 「分数」という統一概念へ統合
\item ΔGED = -1.8 (ノード数が減少)
\item $\mathcal{F}_t = 2.1$ (高い内発報酬)
\end{itemize}

\subsubsection{分裂操作(Split)}

\textbf{例: 積分概念の分裂(ID 13)}
\begin{itemize}
\item 前: 「積分=面積計算」という単一理解
\item 後: 「リーマン積分」と「反微分」に分裂
\item ΔGED = -2.3 (エッジが増えても全体構造は単純化)
\item $\mathcal{F}_t = 2.7$ (非常に高い内発報酬)
\end{itemize}

\subsubsection{再編成操作(Reorganize)}

\textbf{例: 実数体系の再編成(ID 16)}
\begin{itemize}
\item 前: 整数、分数、小数、無理数が別々の概念
\item 後: 「完備順序体」としての統一的理解
\item ΔGED = -3.1 (大規模な構造再編成)
\item $\mathcal{F}_t = 3.5$ (最高レベルの内発報酬)
\end{itemize}

\subsection{G.4 記憶操作の統計的分析}

\begin{table}[H]
\centering
\caption{記憶操作の統計}
\begin{tabular}{|l|c|c|c|}
\hline
操作タイプ & 発生回数 & 平均ΔGED & 平均$\mathcal{F}_t$ \\
\hline
新規追加 & 11 & -0.45 & 0.41 \\
統合 & 2 & -1.95 & 2.30 \\
分裂 & 1 & -2.30 & 2.70 \\
再編成 & 1 & -3.10 & 3.50 \\
拡張 & 3 & -0.67 & 0.60 \\
再定義 & 2 & -1.15 & 1.03 \\
\hline
\end{tabular}
\end{table}

\subsection{G.5 洞察生成のタイミング}

\begin{figure}[H]
\centering
\includegraphics[width=\textwidth]{figures/gedig_score_timeline.png}
\caption{geDIGスコアの時間変化。赤線は洞察検出闾値(2.5)を示す。
3つの主要なピークが記憶再編成イベントに対応。}
\label{fig:gedig_timeline}
\end{figure}

\subsection{G.6 実験2の結論}

数学概念進化実験により、以下の重要な知見が得られた:

\begin{enumerate}
\item \textbf{内発報酬と記憶再編成の相関}: $\mathcal{F}_t > 2.5$で必ず大規模な記憶操作が発生
\item \textbf{段階的学習の重要性}: 概念の段階的発展がより高いΔGED値を生成
\item \textbf{記憶操作のパターン}: 統合→分裂→再編成の順序が人間の概念学習と一致
\item \textbf{構造的進化}: 初期の分散構造から階層的構造への進化
\end{enumerate}

\end{document}