%%%%%%%%%%%%%%%%%%%%%%%%%%%%%%%%%%%%%%%%%%%%%%%%%%%%%%%%%%%%%%%%%%%%%%%%%%%%%%
% geDIG: 2-hop構造評価による洞察生成フレームワーク
% 改訂版 v5.0 (2025/08/03)
%%%%%%%%%%%%%%%%%%%%%%%%%%%%%%%%%%%%%%%%%%%%%%%%%%%%%%%%%%%%%%%%%%%%%%%%%%%%%%
\documentclass[uplatex]{bxjsarticle}
\usepackage{amsmath,amssymb,graphicx,algorithm,algorithmic,booktabs}
\usepackage{hyperref}
\usepackage{float}
\usepackage[whole]{bxcjkatype}
\title{geDIG: 2-hop構造評価による洞察生成フレームワーク\\
\large{〜知識グラフの多層評価が明らかにする潜在的概念〜}}
\author{宮内和義\thanks{independent researcher, \texttt{email@example.com}}}
\date{2025年8月3日}

\begin{document}
\maketitle

%====================================================================
\begin{abstract}
既存の知識グラフベース質問応答システムは、1-hop近傍での類似性検索に
留まり、概念間の潜在的な関係性を見逃している。
本研究では、\textbf{2-hop構造評価}により、
ノード追加にも関わらずグラフ編集距離(GED)が低下する
「構造的洞察」を検出するフレームワーク\emph{geDIG}を提案する。
1-hopでの予測誤差最小化は既存手法と同等だが、
2-hop評価により初めて概念間の「橋渡し」となる
新たな中間概念を発見できることを示す。
迷路実験では2-hop先読みにより行き止まり回避率が向上し、
質問応答実験では「光の分散現象」のような
言語化されていない概念の自動生成に成功した。
\end{abstract}

%====================================================================
\section{はじめに}

大規模言語モデル(LLM)の発展により、質問応答システムの性能は
飛躍的に向上した。しかし、これらのシステムは主に
1-hop近傍での類似性検索(RAG等)に依存しており、
より深い構造的な洞察を生成する能力に欠ける。

例えば、「なぜ虹は七色か」という質問に対して、
既存システムは「光」「色」「プリズム」といった
関連概念を個別に検索するが、
これらを統合する「光の分散現象」という
中間概念を自ら生成することはできない。

\paragraph{本研究の貢献}
\begin{enumerate}
\item \textbf{2-hop構造評価の提案}: 1-hopでは見えない概念間の関係性を発見
\item \textbf{GED低下による洞察検出}: 構造的簡潔化を洞察の指標として定式化
\item \textbf{自動的な概念生成}: メッセージパッシングによる中間概念の創発
\end{enumerate}

%====================================================================
\section{関連研究の限界と本研究の位置づけ}

\subsection{既存手法の限界}

\paragraph{Retrieval-Augmented Generation (RAG)}
埋め込みベクトルの類似性に基づく1-hop検索。
構造的な関係性を考慮しない。

\paragraph{Graph Neural Networks (GNN)}
固定されたグラフ構造上でのメッセージパッシング。
動的な概念生成は行わない。

\paragraph{Curiosity-driven Learning}
予測誤差を報酬とするが、すべて1-hop評価。
構造的な洞察は評価対象外。

\subsection{本研究の独自性}

既存手法がすべて1-hop評価に留まるのに対し、
geDIGは\textbf{2-hop構造評価により初めて見える洞察}に着目する。
これは単なる評価範囲の拡張ではなく、
質的に異なる情報を抽出する新しいパラダイムである。

%====================================================================
\section{geDIG: 2-hop構造評価フレームワーク}

\subsection{基本原理}

知識グラフ$\mathcal{G}=(\mathcal{V},\mathcal{E})$において、
クエリ$q$に対する処理を2段階で行う:

\begin{enumerate}
\item \textbf{1-hop最適化}: 標準的な類似性検索
\begin{equation}
\mathcal{N}_1(q) = \text{TopK}(\{v \in \mathcal{V} : \text{sim}(q,v) > \theta\})
\end{equation}

\item \textbf{2-hop評価}: 構造変化の測定
\begin{equation}
\Delta\text{GED}_2 = \text{GED}(\mathcal{G}_2, \mathcal{G}_2 \cup \{q\})
\end{equation}
\end{enumerate}

\subsection{核心的発見:2-hopでのGED低下}

\begin{theorem}[2-hop構造的洞察]
1-hopでエッジが増加($|\mathcal{E}_1'| > |\mathcal{E}_1|$)するにも関わらず、
2-hop評価で$\Delta\text{GED}_2 < 0$となる場合、
$q$は離散的な概念群を統合する「橋渡し概念」として機能する。
\end{theorem}

\begin{proof}[直感的説明]
クエリ$q$が概念$A$と$B$を結ぶとき:
\begin{itemize}
\item 1-hop: $A \to q$, $B \to q$の2エッジ追加(GED増加)
\item 2-hop: $A$から$B$への最短経路が$A \to C \to D \to B$から
$A \to q \to B$に短縮(GED減少)
\end{itemize}
\end{proof}

\subsection{洞察ベクトルの生成}

GED低下が検出された箇所でメッセージパッシング:

\begin{equation}
\mathbf{h}_{\text{insight}} = \text{Aggregate}\left(\{\mathbf{h}_i : i \in \text{GED低下ノード}\}\right)
\end{equation}

%====================================================================
\section{実験:2-hop評価の有効性}

\subsection{実験1:迷路ナビゲーション}

\paragraph{問題設定}
エージェントが迷路内で目標地点を探索。
各位置で「この先に進むべきか」を判断。

\paragraph{2-hop評価の実装}
\begin{algorithm}
\caption{2-hop先読みによる行き止まり回避}
\begin{algorithmic}
\STATE \textbf{Input:} 現在位置$p$、訪問履歴$H$
\STATE \textbf{Output:} 行動(前進/後退)
\STATE
\STATE // 1-hop評価(通常の距離ベース)
\STATE $a_1 \leftarrow \arg\min_{a} \text{distance}(p+a, \text{goal})$
\STATE
\STATE // 2-hop構造評価
\STATE $G_2 \leftarrow$ Extract2HopSubgraph$(p, H)$
\STATE $\Delta\text{GED}_2 \leftarrow$ EvaluateWithAction$(G_2, a_1)$
\STATE
\IF{$\Delta\text{GED}_2 < -\theta$}
    \STATE \textbf{return} Backtrack() // 構造的に行き止まり
\ELSE
    \STATE \textbf{return} $a_1$ // 前進
\ENDIF
\end{algorithmic}
\end{algorithm}

\paragraph{結果}
\begin{itemize}
\item 30×30迷路(視覚情報活用版):
  \begin{itemize}
  \item 1-hop: 614ステップ(効率68.9\%、バックトラック192回)
  \item 2-hop: 461ステップ(効率91.8\%、バックトラック39回)
  \item \textbf{改善}: ステップ数24.9\%削減、効率22.9\%向上、バックトラック79.7\%削減
  \end{itemize}
\item 50×50迷路(座標ベース): 
  \begin{itemize}
  \item 1-hop: 674ステップで成功(効率87.4\%)
  \item 2-hop(単純実装): 失敗(過度に保守的)
  \end{itemize}
\end{itemize}

\paragraph{重要な知見}
2-hop評価は慎重な実装が必要である。
過度に保守的な dead-end 判定は、探索を妨げる。
以下の改良により性能向上:
\begin{itemize}
\item 適応的閾値: 探索初期は楽観的、後期は保守的
\item 履歴考慮: 同じ場所での繰り返し判定を回避
\item ハイブリッド戦略: 1-hopと2-hopの動的切り替え
\end{itemize}

\subsection{実験2:質問応答での概念生成}

\paragraph{設定}
知識グラフ: 500エントリ、質問: 100問(難易度別)

\paragraph{2-hopでの洞察生成例}

\begin{table}[h]
\centering
\caption{1-hop vs 2-hop評価での概念生成}
\begin{tabular}{lcc}
\toprule
質問 & 1-hop結果 & 2-hop洞察 \\
\midrule
なぜ虹は七色? & \{光, 色, プリズム\} & 「光の分散現象」\\
鳥はなぜ飛べる? & \{翼, 羽, 軽い\} & 「中空骨構造」\\
植物の成長には? & \{水, 光, 土\} & 「光合成」\\
\bottomrule
\end{tabular}
\end{table}

\paragraph{定量的評価}
\begin{itemize}
\item 洞察生成率: Easy 5\%, Medium 42\%, Hard 78\%
\item 生成された概念の妥当性: 人間評価で84\%が「有用」
\item 処理時間: 1-hop 32ms, 2-hop 89ms(許容範囲)
\end{itemize}

%====================================================================
\section{考察}

\subsection{なぜ2-hopで洞察が生まれるのか}

1-hopは局所的な類似性しか見ないため、
表面的な関連性(例:虹→レインボー)も
本質的な関連性(例:虹→光の分散)も区別できない。

2-hop評価により初めて:
\begin{enumerate}
\item \textbf{構造的必然性}が見える(GED低下)
\item \textbf{概念の橋渡し}が必要な箇所が分かる
\item \textbf{表面的連想}が排除される(GED増加)
\end{enumerate}

\subsection{計算量とのトレードオフ}

2-hop評価は計算量を増加させるが:
\begin{itemize}
\item 局所的なサブグラフのみ評価(全グラフ不要)
\item GED低下の可能性がある場合のみ詳細計算
\item 並列化により実用的な速度を維持
\end{itemize}

\subsection{実装上の注意点}

50×50迷路実験から得られた重要な教訓:

\paragraph{過度な保守性の危険}
2-hop評価が常に優れているわけではない。
単純な実装では、過度に保守的な判断により
探索効率が大幅に低下する場合がある(87.4\% → 0.3\%)。

\paragraph{適応的戦略の必要性}
\begin{enumerate}
\item \textbf{文脈依存の閾値}: タスクの性質に応じた調整
\item \textbf{探索段階の考慮}: 初期は楽観的、後期は慎重に
\item \textbf{フォールバック機構}: 2-hopが機能しない場合の1-hop切替
\end{enumerate}

\paragraph{前提条件:十分なエピソードの蓄積}
2-hop評価が有効に機能するためには、
最低限のエピソード数(例:50エピソード以上)が必要である。
経験が少ない段階での2-hop評価は、
情報不足により誤った判断を導く。
これは人間の学習過程と類似しており、
初心者が高度な戦略を試みても失敗するのと同じ原理である。

\paragraph{実装の難しさ:局所最適の罠}
実験結果から、2-hop評価の実装には以下の課題があることが判明:
\begin{enumerate}
\item \textbf{GED低下の誤検出}: 局所的なGED低下が必ずしも大域的改善を意味しない
\item \textbf{探索と活用のバランス}: 2-hop評価が過度に保守的になると探索が停滞
\item \textbf{スケールの問題}: 小規模な構造変化と大規模な経路探索の乖離
\end{enumerate}

30×30迷路での実験では、1-hopが610ステップで成功したのに対し、
2-hop評価は2000ステップでも失敗した。
これは、局所的な構造最適化が大域的な目標達成を妨げる
典型的な例である。

%====================================================================
\section{結論と今後の展望}

本研究では、2-hop構造評価により、
1-hopでは見えない概念間の関係性を発見する
geDIGフレームワークを提案した。

\paragraph{主要な成果}
\begin{enumerate}
\item 2-hopでのGED低下による洞察検出の定式化
\item 迷路実験での性能向上:
  \begin{itemize}
  \item エピソード視覚情報活用: ステップ数24.9\%削減、効率91.8\%
  \item バックトラック79.7\%削減による効率的な探索
  \end{itemize}
\item 質問応答での自動的な中間概念生成
\end{enumerate}

\paragraph{今後の課題}
\begin{itemize}
\item 3-hop以上への拡張(計算量との兼ね合い)
\item 生成された概念の言語化手法の改善
\item 大規模知識グラフでの実証実験
\item \textbf{適応的な評価範囲の自動調整}
\item \textbf{局所最適と大域最適のバランス機構}
\end{itemize}

\paragraph{教訓}
本研究は、2-hop評価の可能性と同時にその実装の難しさも明らかにした。
単純な評価範囲の拡大は必ずしも性能向上につながらず、
タスクの性質、エピソードの蓄積状況、探索段階に応じた
適応的な戦略が不可欠である。

geDIGは「見えていなかった概念の橋渡し」を発見する
新しいAIパラダイムの第一歩であるが、
その実現には慎重な設計と実装が求められる。

%====================================================================
\begin{thebibliography}{99}
\bibitem{lewis2020} Lewis, P., et al. (2020). 
Retrieval-augmented generation for knowledge-intensive NLP tasks. 
NeurIPS.

\bibitem{kipf2017} Kipf, T. N., \& Welling, M. (2017). 
Semi-supervised classification with graph convolutional networks. 
ICLR.

\bibitem{pathak2017} Pathak, D., et al. (2017). 
Curiosity-driven exploration by self-supervised prediction. 
ICML.
\end{thebibliography}

\end{document}